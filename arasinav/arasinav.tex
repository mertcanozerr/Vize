% Options for packages loaded elsewhere
\PassOptionsToPackage{unicode}{hyperref}
\PassOptionsToPackage{hyphens}{url}
%
\documentclass[
]{article}
\usepackage{amsmath,amssymb}
\usepackage{lmodern}
\usepackage{iftex}
\ifPDFTeX
  \usepackage[T1]{fontenc}
  \usepackage[utf8]{inputenc}
  \usepackage{textcomp} % provide euro and other symbols
\else % if luatex or xetex
  \usepackage{unicode-math}
  \defaultfontfeatures{Scale=MatchLowercase}
  \defaultfontfeatures[\rmfamily]{Ligatures=TeX,Scale=1}
\fi
% Use upquote if available, for straight quotes in verbatim environments
\IfFileExists{upquote.sty}{\usepackage{upquote}}{}
\IfFileExists{microtype.sty}{% use microtype if available
  \usepackage[]{microtype}
  \UseMicrotypeSet[protrusion]{basicmath} % disable protrusion for tt fonts
}{}
\makeatletter
\@ifundefined{KOMAClassName}{% if non-KOMA class
  \IfFileExists{parskip.sty}{%
    \usepackage{parskip}
  }{% else
    \setlength{\parindent}{0pt}
    \setlength{\parskip}{6pt plus 2pt minus 1pt}}
}{% if KOMA class
  \KOMAoptions{parskip=half}}
\makeatother
\usepackage{xcolor}
\usepackage[margin=1in]{geometry}
\usepackage{graphicx}
\makeatletter
\def\maxwidth{\ifdim\Gin@nat@width>\linewidth\linewidth\else\Gin@nat@width\fi}
\def\maxheight{\ifdim\Gin@nat@height>\textheight\textheight\else\Gin@nat@height\fi}
\makeatother
% Scale images if necessary, so that they will not overflow the page
% margins by default, and it is still possible to overwrite the defaults
% using explicit options in \includegraphics[width, height, ...]{}
\setkeys{Gin}{width=\maxwidth,height=\maxheight,keepaspectratio}
% Set default figure placement to htbp
\makeatletter
\def\fps@figure{htbp}
\makeatother
\setlength{\emergencystretch}{3em} % prevent overfull lines
\providecommand{\tightlist}{%
  \setlength{\itemsep}{0pt}\setlength{\parskip}{0pt}}
\setcounter{secnumdepth}{-\maxdimen} % remove section numbering
\newlength{\cslhangindent}
\setlength{\cslhangindent}{1.5em}
\newlength{\csllabelwidth}
\setlength{\csllabelwidth}{3em}
\newlength{\cslentryspacingunit} % times entry-spacing
\setlength{\cslentryspacingunit}{\parskip}
\newenvironment{CSLReferences}[2] % #1 hanging-ident, #2 entry spacing
 {% don't indent paragraphs
  \setlength{\parindent}{0pt}
  % turn on hanging indent if param 1 is 1
  \ifodd #1
  \let\oldpar\par
  \def\par{\hangindent=\cslhangindent\oldpar}
  \fi
  % set entry spacing
  \setlength{\parskip}{#2\cslentryspacingunit}
 }%
 {}
\usepackage{calc}
\newcommand{\CSLBlock}[1]{#1\hfill\break}
\newcommand{\CSLLeftMargin}[1]{\parbox[t]{\csllabelwidth}{#1}}
\newcommand{\CSLRightInline}[1]{\parbox[t]{\linewidth - \csllabelwidth}{#1}\break}
\newcommand{\CSLIndent}[1]{\hspace{\cslhangindent}#1}
\ifLuaTeX
  \usepackage{selnolig}  % disable illegal ligatures
\fi
\IfFileExists{bookmark.sty}{\usepackage{bookmark}}{\usepackage{hyperref}}
\IfFileExists{xurl.sty}{\usepackage{xurl}}{} % add URL line breaks if available
\urlstyle{same} % disable monospaced font for URLs
\hypersetup{
  hidelinks,
  pdfcreator={LaTeX via pandoc}}

\author{}
\date{\vspace{-2.5em}}

\begin{document}

title: ``Myanmar'daki Süpermarket Müşterilerinin Puanlamaları Üzerine
Tahmin'' author: - Mert Can Özer\footnote{19080238} bibliography:
../bibliography/biblio.bib csl: ../csl/apa-tr.csl header-includes: -

\usepackage{polyglossia}

\begin{itemize}
\item
  \setmainlanguage{turkish}

  \begin{itemize}
  \item
    \usepackage{booktabs}
  \item
    \usepackage{caption}
  \item
    \captionsetup[table]{skip=10pt}

    output: bookdown::pdf\_document2: fig\_caption: yes fig\_height: 3
    fig\_width: 4 keep\_tex: no latex\_engine: xelatex number\_sections:
    yes toc: no geometry: margin=1in link-citations: yes urlcolor: blue
    fontsize: 12pt biblio-style: apalike ---
  \end{itemize}
\end{itemize}

\hypertarget{giriux15f}{%
\section{Giriş}\label{giriux15f}}

Myanmar Birliği Cumhuriyeti veya diğer bilinen adlarıyla Myanmar, Burma,
Birmanya Güneydoğu Asya'da bir ülkedir.55 milyona yakın nüfusu ile
Güneydoğu Asya anakarasındakien büyük ülke olan Myanmar, 1 milyonu aşkın
3, toplamda 78 şehre sahiptir (@Wikipedia). Kökeni 9. yüzyıla kadar
dayanan bu ülke gerek konumu gerekse üzerinde uzun yıllar boyunca
uygulanan stratejiler sayesinde birçok alanda kendi adından
bahsettirmiştir.

2.Dünya Savaşı sonuna kadar İngiliz sömürgesiolarak Birmanya adıyla
tarih sahnesinde yer alan Myanmar 1948'de sömürgeden çıkarak pek çok
alanda olduğu gibi ekonomik özgürlüğünü de kazanmış oldu (@Wikipedia).
İç karışıklıkların çok sık olduğu ülkede gerçekleşen çalkantılar
günümüzde bile halen bir ekonomik istikrar yakalayamamasının en büyük
sebeplerinden birisidir.

Myanmar, etnik çeşitliliği yüksek olan bir bölge olması sebebiyle her
alanda yönelimlerin de çeşitlilik kazanması durumu söz konusudur. Bu
çalışmada Myanmar'ın en büyük 3 şehri olan Naypitaw (Nepido), Mandalay
ve Yangon'da bulunan alışveriş merkezi ve buradaki müşterilerin
değerlendirmeleri üzerine hazırlanmış veri seti üzerine
çalışılmıştır.Bahsi geçen 3 şehirdeki müşteriler, 2019 yılında çeşitli
değişkenler ışığında incelenerek ortaya 1000 gözleem konulmuştur.

\hypertarget{uxe7alux131ux15fmanux131n-amacux131}{%
\subsection{Çalışmanın
Amacı}\label{uxe7alux131ux15fmanux131n-amacux131}}

Myanmar'da 2019 yılında 3 büyük şehirde gerçekleştirilmiş olan market
alışverişleri ve müşteri puanlamaları incelenerek 2020-2021 yılları
puanlamaları öngörülecektir. Bu çalışmadan elde edilen bulguların,
Myanmar'daki müşterilerin eğilimlerine göre süpermarketlerin satış
politikalarına yardımcı olacağı düşünülmektedir. Ek olarak nüfusa ve
bulunan coğrafi konumlara göre göz önünde bulundurulabilecek bilgilere
de yer verilecektir.

\hypertarget{literatuxfcr}{%
\subsection{Literatür}\label{literatuxfcr}}

Literatür taraması sonucunda müşteri memnuniyetini etkileyen çeşitli
faktörlerin ele alındığı çalışmalara çokça rastlanılmasına karşın; konu
Myanmar özeline geldiğinde çalışmaların sıklığı fazlasıyla seyreliyor.
Çalışma kapsamında çeşitli dergilerden , makalelerden ve raporlardan
yararlanılmıştır.

(@Singh:1998) ve (@Brady) çalışmalarında müşteri odaklı politikaların
pozitif yanlarını ele almıştır. (@Eskiler) ise müşterilerin satın alma
eğilimlerinin müşteri memnuniyetiyle olan ilişkisini açıklamıştır.

\newpage

\hypertarget{references}{%
\section{Kaynakça}\label{references}}

\hypertarget{refs}{}
\begin{CSLReferences}{0}{0}
\end{CSLReferences}

\end{document}
